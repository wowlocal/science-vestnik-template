% ========================================
% ПРОСТОЙ ПРИМЕР ДЛЯ НАЧИНАЮЩИХ
% ========================================
% Это упрощённая версия статьи для обучения
% Используйте как основу для вашей первой статьи

\documentclass[12pt]{article}
\usepackage{vestnik}

\begin{document}

% ========================================
% РУССКАЯ ВЕРСИЯ
% ========================================

\udc{930.85}

\articletitleru{Влияние конфуцианства на современное образование в Китае}

\authorru{А.С. Иванова\footnote{
    \textbf{Иванова Анна Сергеевна} — кандидат исторических наук, 
    доцент кафедры истории Китая, Институт стран Азии и Африки 
    МГУ имени М.В. Ломоносова. Email: ivanova@iaas.msu.ru
}}

\institution{Московский государственный университет имени М.В. Ломоносова, 
Ленинские горы, д. 1, стр. 13, Москва, 119991, Россия}

\abstractru{В статье рассматривается влияние конфуцианских ценностей 
на систему образования в современном Китае. Анализируются основные 
принципы конфуцианской педагогики и их трансформация в условиях 
модернизации китайского общества. Особое внимание уделяется вопросам 
сохранения традиционных ценностей в образовательном процессе.}

\keywordsru{Китай, конфуцианство, образование, традиция, модернизация}

\forcitation{Иванова А.С. Влияние конфуцианства на современное образование в Китае // Вестн. Моск. ун-та. Сер. 13. Востоковедение. 2025. Т. XX, № X. С. 00--00. DOI: 10.55959/MSU0320-8095-13-00-0-0.}

\vspace{1em}

% ========================================
% АНГЛИЙСКАЯ ВЕРСИЯ
% ========================================

\articletitleen{The Influence of Confucianism on Modern Education in China}

\authoren{Anna S. Ivanova}

\institutionen{Lomonosov Moscow State University, 
Leninskie Gory, 1, bld. 13, Moscow, 119991, Russia}

\aboutauthor{\textbf{Anna S. Ivanova} — PhD in History, 
Associate Professor, Institute of Asian and African Studies, 
Lomonosov Moscow State University. Email: ivanova@iaas.msu.ru}

\abstracten{The article examines the influence of Confucian values on the education system in modern China. The main principles of Confucian pedagogy and their transformation in the context of Chinese society modernization are analyzed. Special attention is paid to preserving traditional values in the educational process.}

\keywordsen{China, Confucianism, education, tradition, modernization}

\forcitationen{Ivanova A.S. The Influence of Confucianism on Modern Education in China. Lomonosov Oriental Studies Journal, 2025, Vol. XX, No. X, pp. 00--00. DOI: 10.55959/MSU0320-8095-13-00-0-0 (In Russ.).}

\vspace{1em}

% ========================================
% ОСНОВНОЙ ТЕКСТ
% ========================================

\section{Введение}

Проблема сохранения традиционных ценностей в условиях 
модернизации общества актуальна для многих стран\footnote{Иванов И.И. 
Традиция и современность. М.: Наука, 2020. С. 45.}. Особенно это 
касается Китая, где конфуцианство на протяжении двух тысячелетий 
определяло основы образования и воспитания.

Современная система образования КНР представляет собой уникальный 
синтез традиционных и современных подходов\footnote{Петров П.П. 
Образовательные реформы в Китае // Востоковедение. 2021. № 2. 
С. 10--25.}. Этот синтез заслуживает детального изучения.

\section{Основные принципы конфуцианской педагогики}

Конфуцианская традиция выделяет несколько ключевых принципов образования:
\begin{itemize}
\item Приоритет морального воспитания
\item Уважение к учителю
\item Постоянное самосовершенствование
\item Связь обучения с практикой
\end{itemize}

Эти принципы, сформулированные более двух тысяч лет назад, 
продолжают влиять на современную образовательную систему Китая.

\subsection{Роль учителя}

В конфуцианской традиции учитель занимает особое место. 
Как отмечал сам Конфуций, <<учитель открывает дверь, 
но войти должен сам ученик>>.

\section{Трансформация традиций в современных условиях}

XX век принёс значительные изменения в китайскую систему образования. 
Однако многие традиционные ценности не только сохранились, 
но и получили новое развитие.

Современные китайские педагоги стремятся совместить лучшие достижения 
мировой педагогической мысли с традиционными конфуцианскими принципами.

\section{Заключение}

Опыт Китая показывает, что традиционные ценности могут успешно 
сосуществовать с современными образовательными технологиями. 
Конфуцианские принципы, адаптированные к условиям XXI века, 
продолжают играть важную роль в формировании образовательной 
системы КНР.

% ========================================
% СПИСОК ЛИТЕРАТУРЫ
% ========================================

\bibliographyru

\bibliosectioncyrillic
\begin{bibliolist}
\item Иванов И.И. Традиция и современность в образовании. М.: Наука, 2020. 300 с.
\item Петров П.П. Образовательные реформы в Китае // Востоковедение. 2021. № 2. С. 10--25.
\end{bibliolist}

\bibliosectionlatin
\begin{bibliolist}
\item Chen L. Confucianism and Modern Education. Beijing: Academic Press, 2019. 250 p.
\item Smith J. Traditional Values in Contemporary China // Asian Studies. 2020. Vol. 15. P. 30--45.
\end{bibliolist}

% ========================================
% REFERENCES
% ========================================

\referencesen

\begin{bibliolist}
\item Ivanov I.I. \textit{Traditsiya i sovremennost' v obrazovanii} [Tradition and Modernity in Education]. Moscow, Nauka Publ., 2020, 300 p. (In Russ.)
\item Petrov P.P. Obrazovatel'nye reformy v Kitae [Educational Reforms in China]. \textit{Vostokovedenie} [Oriental Studies], 2021, no. 2, pp. 10--25. (In Russ.)
\item Chen L. \textit{Confucianism and Modern Education}. Beijing, Academic Press, 2019, 250 p.
\item Smith J. Traditional Values in Contemporary China. \textit{Asian Studies}, 2020, vol. 15, pp. 30--45.
\end{bibliolist}

\end{document}
