% ========================================
% ПРИМЕР ИСПОЛЬЗОВАНИЯ .BIB ФАЙЛА
% ========================================
% Этот файл демонстрирует использование библиографической базы данных
% через пакет biblatex для автоматической генерации списка литературы

\documentclass[12pt]{article}

% ========================================
% НАСТРОЙКА BIBLATEX
% ========================================
% Подключаем biblatex ДО пакета vestnik
% Стиль: numeric - нумерованные ссылки [1], [2] и т.д.
% Сортировка: nyt - по имени, году, названию
% Опции: autolang=other - поддержка многоязычности

\usepackage[
    backend=biber,          % Используем biber (современный движок)
    style=numeric,          % Числовой стиль цитирования [1], [2]
    sorting=nyt,            % Сортировка: name, year, title
    maxbibnames=99,         % Показывать всех авторов
    bibencoding=utf8,       % Кодировка UTF-8
    autolang=other,         % Автоопределение языка
]{biblatex}

% Указываем файл с библиографией
\addbibresource{references.bib}

% Теперь подключаем пакет vestnik
\usepackage{vestnik}

\begin{document}

% ========================================
% ЗАГОЛОВОЧНАЯ ЧАСТЬ СТАТЬИ (РУССКИЙ)
% ========================================

\udc{000.00}

\articletitleru{Пример использования .bib файла}

\authorru{И.О. Фамилия\footnote{%
    \textbf{Фамилия Имя Отчество} — кандидат наук, должность, Название организации. Email: example@email.com%
}}

\institution{Название организации, адрес организации}

\abstractru{Эта статья демонстрирует использование файла references.bib для автоматической генерации списка литературы с помощью пакета biblatex. Библиографические ссылки вставляются автоматически, а список формируется на основе цитируемых источников.}

\keywordsru{biblatex, библиография, .bib файл, автоматизация, LaTeX}

\forcitation{Фамилия И.О. Пример использования .bib файла // Вестн. Моск. ун-та. Сер. 13. Востоковедение. 2025. Т. XX, № X. С. 00--00. DOI: 10.55959/MSU0320-8095-13-00-0-0.}

\vspace{1em}

% ========================================
% ЗАГОЛОВОЧНАЯ ЧАСТЬ СТАТЬИ (АНГЛИЙСКИЙ)
% ========================================

\articletitleen{Example of Using .bib File}

\authoren{First M. Last}

\institutionen{Institution Name, Institution Address}

\aboutauthor{\textbf{First Middle Last} — PhD, Position, Institution Name. Email: example@email.com}

\abstracten{This article demonstrates the use of references.bib file for automatic bibliography generation using biblatex package. Bibliographic citations are inserted automatically, and the reference list is formed based on cited sources.}

\keywordsen{biblatex, bibliography, .bib file, automation, LaTeX}

\forcitationen{Last F.M. Example of Using .bib File. Lomonosov Oriental Studies Journal, 2025, Vol. XX, No. X, pp. 00--00. DOI: 10.55959/MSU0320-8095-13-00-0-0 (In Russ.).}

\vspace{1em}

% ========================================
% ОСНОВНОЙ ТЕКСТ СТАТЬИ
% ========================================

\section{Введение}

При использовании biblatex ссылки на источники вставляются с помощью команды \texttt{\textbackslash cite\{\}}. Например, можно сослаться на книгу Иванова \cite{ivanov2020} или на статью Петрова \cite{petrov2021}.

Также можно указывать конкретные страницы: \cite[с. 25]{smith2019} или \cite[p. 50]{johnson2020}.

\section{Основная часть}

В работе \cite{sidorov2022} рассматриваются современные подходы. Зарубежные исследования \cite{brown2021} дополняют картину.

Множественные ссылки вставляются так: \cite{ivanov2020, petrov2021, smith2019}.

\subsection{Подраздел}

Онлайн-ресурсы также можно цитировать \cite{webresource2023}, как и диссертации \cite{nikolaev2018}.

\section{Заключение}

Использование .bib файла значительно упрощает работу с библиографией, особенно в больших статьях с множеством источников.

% ========================================
% СПИСОК ЛИТЕРАТУРЫ (АВТОМАТИЧЕСКИ)
% ========================================
% При использовании biblatex список литературы генерируется автоматически
% командой \printbibliography

\printbibliography[title={\centering\textbf{Список литературы}}]

% ========================================
% ПРИМЕЧАНИЯ
% ========================================
%
% КОМПИЛЯЦИЯ:
% Для корректной работы с biblatex нужно выполнить последовательность команд:
% 1. pdflatex article-bibfile.tex
% 2. biber article-bibfile
% 3. pdflatex article-bibfile.tex
% 4. pdflatex article-bibfile.tex
%
% Или использовать latexmk для автоматической компиляции:
% latexmk -pdf -pdflatex="pdflatex -interaction=nonstopmode" article-bibfile.tex
%
% РАЗДЕЛЕНИЕ ПО ЯЗЫКАМ:
% Для разделения списка литературы на кириллицу/латиницу можно использовать
% фильтры biblatex (см. документацию biblatex или спросите у автора шаблона)

\end{document}
