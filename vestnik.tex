% ========================================
% КЛАСС ДОКУМЕНТА
% ========================================
% Требование: кегль 12 для основного текста
\documentclass[12pt]{article}    % article - класс для статей, 12pt - размер шрифта

% ========================================
% ПОДДЕРЖКА РУССКОГО ЯЗЫКА
% ========================================
% Требование: поддержка русского языка для текста статьи
\usepackage[utf8]{inputenc}      % Кодировка UTF-8 для входных файлов
\usepackage[T2A]{fontenc}        % Кодировка T2A для кириллических шрифтов
\usepackage[russian]{babel}      % Правила переносов и локализация для русского языка

% ========================================
% ПОЛЯ СТРАНИЦЫ
% ========================================
% Требование: настройка полей документа
\usepackage{geometry}
\geometry{margin=1in}            % Поля 1 дюйм (2.54 см) со всех сторон

% ========================================
% ШРИФТ ОСНОВНОГО ТЕКСТА
% ========================================
% Требование: Times New Roman, кегль 12, интервал 1,5
\usepackage{newtxtext}           % Шрифт Times New Roman для основного текста

% ========================================
% МЕЖСТРОЧНЫЙ ИНТЕРВАЛ
% ========================================
% Требование: интервал 1,5 для основного текста
\usepackage{setspace}
\onehalfspacing                  % Устанавливает межстрочный интервал 1.5

% ========================================
% АБЗАЦНЫЙ ОТСТУП
% ========================================
% Требование: абзацный отступ 1,25 см
\setlength{\parindent}{1.25cm}   % Устанавливает отступ первой строки абзаца 1.25 см

% ========================================
% НАСТРОЙКА СНОСОК
% ========================================
% Требование: Times New Roman 10pt, интервал 1
\usepackage[bottom]{footmisc}    % Размещает сноски внизу страницы
% Настройка размера шрифта сносок (10pt) и одинарного интервала
\makeatletter
\renewcommand\@makefntext[1]{%
    \setlength{\parindent}{1.25cm}%  % Сохраняем абзацный отступ в сносках
    \noindent
    \makebox[1.8em][l]{\@thefnmark}% Номер сноски
    \fontsize{10}{12}\selectfont     % Шрифт 10pt с интервалом 1.0 (12pt базовая линия)
    #1%
}
\makeatother

% ========================================
% НУМЕРАЦИЯ СТРАНИЦ
% ========================================
% Требование: все страницы рукописи нумеруются, включая список литературы
\pagestyle{plain}                % Стиль страницы с номером внизу по центру

% ========================================
% МАКРОСЫ ДЛЯ СТРУКТУРЫ СТАТЬИ
% ========================================
% Требование: создание всех обязательных элементов статьи по требованиям журнала

% УДК (универсальная десятичная классификация)
\newcommand{\udc}[1]{%
    \noindent\textbf{УДК #1}%
    \par\vspace{0.5em}%
}

% Заголовок статьи на русском (полужирный, прописные, 14pt)
\newcommand{\articletitleru}[1]{%
    \begin{center}
        \fontsize{14}{16}\selectfont\bfseries\MakeUppercase{#1}
    \end{center}
    \vspace{0.3em}
}

% Автор(ы) на русском (полужирный, 14pt)
\newcommand{\authorru}[1]{%
    \begin{center}
        \fontsize{14}{16}\selectfont\bfseries #1
    \end{center}
    \vspace{0.3em}
}

% Организация и адрес (курсив, 12pt)
\newcommand{\institution}[1]{%
    \begin{center}
        \fontsize{12}{14}\selectfont\itshape #1
    \end{center}
    \vspace{0.5em}
}

% Аннотация на русском (полужирный + курсив, 200-300 слов)
\newcommand{\abstractru}[1]{%
    \noindent\textbf{\textit{Аннотация:}} \textit{#1}%
    \par\vspace{0.5em}%
}

% Ключевые слова на русском (полужирный + курсив, 5-7 слов)
\newcommand{\keywordsru}[1]{%
    \noindent\textbf{\textit{Ключевые слова:}} \textit{#1}%
    \par\vspace{0.5em}%
}

% Финансирование (опционально)
\newcommand{\funding}[1]{%
    \noindent\textbf{Финансирование:} #1%
    \par\vspace{0.5em}%
}

% Блок "Для цитирования"
\newcommand{\citation}[1]{%
    \noindent\textbf{Для цитирования:} #1%
    \par\vspace{1em}%
}

% ========================================
% МАКРОСЫ ДЛЯ АНГЛИЙСКОЙ ВЕРСИИ
% ========================================
% Требование: английские версии всех элементов статьи

% Заголовок статьи на английском (bold, uppercase, 14pt)
\newcommand{\articletitleen}[1]{%
    \begin{center}
        \fontsize{14}{16}\selectfont\bfseries\MakeUppercase{#1}
    \end{center}
    \vspace{0.3em}
}

% Автор(ы) на английском (bold, 14pt)
% Формат: First name, Middle initial, Last name
\newcommand{\authoren}[1]{%
    \begin{center}
        \fontsize{14}{16}\selectfont\bfseries #1
    \end{center}
    \vspace{0.3em}
}

% Организация и адрес на английском (italic, 12pt)
\newcommand{\institutionen}[1]{%
    \begin{center}
        \fontsize{12}{14}\selectfont\itshape #1
    \end{center}
    \vspace{0.5em}
}

% About the author(s) - информация об авторах на английском
\newcommand{\aboutauthor}[1]{%
    \noindent\textbf{About the author:} #1%
    \par\vspace{0.5em}%
}

% Abstract на английском (bold, italic, 200-300 words)
\newcommand{\abstracten}[1]{%
    \noindent\textbf{\textit{Abstract:}} \textit{#1}%
    \par\vspace{0.5em}%
}

% Key words на английском (bold, italic, 5-7)
\newcommand{\keywordsen}[1]{%
    \noindent\textbf{\textit{Key words:}} \textit{#1}%
    \par\vspace{0.5em}%
}

% Финансирование на английском (опционально)
\newcommand{\fundingen}[1]{%
    \noindent\textbf{Funding:} #1%
    \par\vspace{0.5em}%
}

% Блок "For citation" на английском
\newcommand{\citationen}[1]{%
    \noindent\textbf{For citation:} #1%
    \par\vspace{1em}%
}

% ========================================
% МАКРОСЫ ДЛЯ БИБЛИОГРАФИИ
% ========================================
% Требование: список литературы и References с правильным форматированием

% Начало раздела "Список литературы" (полужирный заголовок)
\newcommand{\bibliographyru}{%
    \section*{\centering\textbf{Список литературы}}%
    \addcontentsline{toc}{section}{Список литературы}%
}

% Подразделы списка литературы
\newcommand{\bibliosectioncyrillic}{%
    \subsection*{На русском языке (кириллица)}%
}

\newcommand{\bibliosectionlatin}{%
    \subsection*{На западных языках (латиница)}%
}

\newcommand{\bibliosectionoriental}{%
    \subsection*{На восточных языках (иероглифика)}%
}

% Начало раздела "References" (полужирный заголовок)
\newcommand{\referencesen}{%
    \section*{\centering\textbf{References}}%
    \addcontentsline{toc}{section}{References}%
}

% Окружение для списка литературы без номеров
\newenvironment{bibliolist}{%
    \begin{list}{}{%
        \setlength{\leftmargin}{0pt}%
        \setlength{\itemindent}{-1.5em}%
        \setlength{\itemsep}{0.5em}%
        \setlength{\parsep}{0pt}%
    }%
}{%
    \end{list}%
}

% ========================================
% ИНФОРМАЦИЯ О ДОКУМЕНТЕ (Пример заполнения)
% ========================================
% Здесь указываются данные для конкретной статьи
% После заполнения эти команды будут использованы в \maketitle

\title{Название статьи}
\author{Имя автора}
\date{\today}

\begin{document}

% ========================================
% ЗАГОЛОВОЧНАЯ ЧАСТЬ СТАТЬИ (РУССКИЙ)
% ========================================

% УДК - укажите код классификации
\udc{000.00}

% Заголовок статьи на русском (полужирный, прописные, 14pt)
\articletitleru{Название статьи}

% ФИО автора(ов) на русском (полужирный, 14pt)
\authorru{И.О. Фамилия\footnote{%
    \textbf{Фамилия Имя Отчество} — кандидат наук, должность, Название организации. Email: example@email.com%
}}

% Организация и адрес (курсив, 12pt)
\institution{Название организации, адрес организации}

% Аннотация (полужирный + курсив, 200-300 слов)
\abstractru{Здесь размещается аннотация объемом 200--300 слов. Аннотация должна кратко излагать содержание статьи, основные результаты исследования и выводы. Текст аннотации должен быть структурированным и информативным, позволяя читателю понять суть работы без обращения к полному тексту.}

% Ключевые слова (полужирный + курсив, 5-7 слов)
\keywordsru{ключевое слово 1, ключевое слово 2, ключевое слово 3, ключевое слово 4, ключевое слово 5}

% Финансирование (опционально, раскомментируйте при необходимости)
% \funding{Исследование выполнено при поддержке гранта...}

% Для цитирования с DOI
\citation{Фамилия И.О. Название статьи // Вестн. Моск. ун-та. Сер. 13. Востоковедение. 2025. Т. XX, № X. С. 00--00. DOI: 10.55959/MSU0320-8095-13-00-0-0.}

\vspace{1em}

% ========================================
% ЗАГОЛОВОЧНАЯ ЧАСТЬ СТАТЬИ (АНГЛИЙСКИЙ)
% ========================================

% Заголовок статьи на английском (bold, uppercase, 14pt)
\articletitleen{Article Title}

% Автор(ы) на английском (bold, 14pt)
% Формат: First name, Middle initial, Last name
\authoren{First M. Last}

% Организация и адрес на английском (italic, 12pt)
\institutionen{Institution Name, Institution Address}

% About the author - информация об авторах
\aboutauthor{\textbf{First Middle Last} — PhD, Position, Institution Name. Email: example@email.com}

% Abstract (bold, italic, 200-300 words)
\abstracten{This is the abstract in English with 200--300 words. The abstract should briefly present the content of the article, main research results and conclusions. The text should be structured and informative, allowing readers to understand the essence of the work without referring to the full text.}

% Key words (bold, italic, 5-7)
\keywordsen{keyword 1, keyword 2, keyword 3, keyword 4, keyword 5}

% Funding (опционально, раскомментируйте при необходимости)
% \fundingen{This research was supported by...}

% For citation с DOI
\citationen{Last F.M. Article Title. Lomonosov Oriental Studies Journal, 2025, Vol. XX, No. X, pp. 00--00. DOI: 10.55959/MSU0320-8095-13-00-0-0 (In Russ.).}

\vspace{1em}

% ========================================
% ОСНОВНОЙ ТЕКСТ СТАТЬИ
% ========================================
% Требование: разделы с подзаголовками (допускаются)
% Кегль 12, Times New Roman, интервал 1.5

\section{Введение}

Текст введения.

\section{Основная часть}

Основной текст статьи.

\subsection{Подраздел}

Более детальное содержание в подразделе.

\section{Заключение}

Текст заключения.

% ========================================
% СПИСОК ЛИТЕРАТУРЫ
% ========================================
% Требование: подстрочные примечания + список литературы (полужирный заголовок)
% Порядок: кириллица → латиница → иероглифика (алфавитный внутри каждой группы)

\bibliographyru

% Раздел на русском языке (кириллица)
\bibliosectioncyrillic
\begin{bibliolist}
\item Иванов И.И. Название книги. М.: Издательство, 2020. 200 с.
\item Петров П.П. Название статьи // Название журнала. 2021. № 1. С. 10--20.
\end{bibliolist}

% Раздел на западных языках (латиница)
\bibliosectionlatin
\begin{bibliolist}
\item Smith J. Book Title. New York: Publisher, 2019. 300 p.
\item Johnson M. Article Title // Journal Name. 2020. Vol. 10. No. 2. P. 50--60.
\end{bibliolist}

% Раздел на восточных языках (иероглифика) - опционально
% \bibliosectionoriental
% \begin{bibliolist}
% \item Транслитерация названия кириллицей [Перевод на русский]. Выходные данные.
% \end{bibliolist}

% ========================================
% REFERENCES (АНГЛИЙСКАЯ ВЕРСИЯ СПИСКА ЛИТЕРАТУРЫ)
% ========================================
% Требование: транслитерация + перевод в квадратных скобках + указание языка

\referencesen

\begin{bibliolist}
\item Ivanov I.I. Nazvanie knigi [Book Title]. Moscow, Izdatel'stvo Publ., 2020, 200 p. (In Russ.)
\item Petrov P.P. Nazvanie stat'i [Article Title]. Nazvanie zhurnala [Journal Name], 2021, no. 1, pp. 10--20. (In Russ.)
\item Smith J. Book Title. New York, Publisher, 2019, 300 p.
\item Johnson M. Article Title. Journal Name, 2020, vol. 10, no. 2, pp. 50--60.
\end{bibliolist}

\end{document}