% ========================================
% КЛАСС ДОКУМЕНТА
% ========================================
% Требование: кегль 12 для основного текста
\documentclass[12pt]{article}    % article - класс для статей, 12pt - размер шрифта

% ========================================
% ПОДДЕРЖКА РУССКОГО ЯЗЫКА
% ========================================
% Требование: поддержка русского языка для текста статьи
\usepackage[utf8]{inputenc}      % Кодировка UTF-8 для входных файлов
\usepackage[T2A]{fontenc}        % Кодировка T2A для кириллических шрифтов
\usepackage[russian]{babel}      % Правила переносов и локализация для русского языка

% ========================================
% ПОЛЯ СТРАНИЦЫ
% ========================================
% Требование: настройка полей документа
\usepackage{geometry}
\geometry{margin=1in}            % Поля 1 дюйм (2.54 см) со всех сторон

% ========================================
% ШРИФТ ОСНОВНОГО ТЕКСТА
% ========================================
% Требование: Times New Roman, кегль 12, интервал 1,5
\usepackage{newtxtext}           % Шрифт Times New Roman для основного текста

% ========================================
% МЕЖСТРОЧНЫЙ ИНТЕРВАЛ
% ========================================
% Требование: интервал 1,5 для основного текста
\usepackage{setspace}
\onehalfspacing                  % Устанавливает межстрочный интервал 1.5

% ========================================
% АБЗАЦНЫЙ ОТСТУП
% ========================================
% Требование: абзацный отступ 1,25 см
\setlength{\parindent}{1.25cm}   % Устанавливает отступ первой строки абзаца 1.25 см

% ========================================
% НАСТРОЙКА СНОСОК
% ========================================
% Требование: Times New Roman 10pt, интервал 1
\usepackage[bottom]{footmisc}    % Размещает сноски внизу страницы
% Настройка размера шрифта сносок (10pt) и одинарного интервала
\makeatletter
\renewcommand\@makefntext[1]{%
    \setlength{\parindent}{1.25cm}%  % Сохраняем абзацный отступ в сносках
    \noindent
    \makebox[1.8em][l]{\@thefnmark}% Номер сноски
    \fontsize{10}{12}\selectfont     % Шрифт 10pt с интервалом 1.0 (12pt базовая линия)
    #1%
}
\makeatother

% ========================================
% ИНФОРМАЦИЯ О ДОКУМЕНТЕ
% ========================================
% ПРИМЕЧАНИЕ: Текущая структура упрощена и НЕ полностью соответствует требованиям
% Требования журнала включают:
% - УДК
% - Название (полужирный, прописные, 14pt)
% - ФИО автора (полужирный, 14pt)
% - Организация (курсив, 12pt) + адрес
% - Информация об авторах в сноске
% - Аннотация (полужирный + курсив, 200-300 слов)
% - Ключевые слова (полужирный + курсив, 5-7)
% - Финансирование
% - Для цитирования с DOI
% - Английские версии всех элементов
\title{Название статьи}
\author{Имя автора}
\date{\today}

\begin{document}

\maketitle

% ========================================
% АННОТАЦИЯ
% ========================================
% Требование: полужирный + курсив, 200-300 слов
% ТЕКУЩАЯ РЕАЛИЗАЦИЯ: стандартное форматирование (НЕ соответствует требованиям)
\begin{abstract}
Здесь размещается аннотация. Это краткое изложение содержания статьи.
\end{abstract}

% ========================================
% ОСНОВНОЙ ТЕКСТ СТАТЬИ
% ========================================
% Требование: разделы с подзаголовками (допускаются)
% Кегль 12, Times New Roman, интервал 1.5

\section{Введение}

Текст введения.

\section{Основная часть}

Основной текст статьи.

\subsection{Подраздел}

Более детальное содержание в подразделе.

\section{Заключение}

Текст заключения.

% ========================================
% СПИСОК ЛИТЕРАТУРЫ
% ========================================
% Требование: подстрочные примечания + список литературы (полужирный заголовок)
% Порядок: кириллица → латиница → иероглифика (алфавитный внутри каждой группы)
% ТЕКУЩАЯ РЕАЛИЗАЦИЯ: отсутствует

% ========================================
% АНГЛИЙСКАЯ ВЕРСИЯ
% ========================================
% Требование: Title, Author, Institution, Abstract, Keywords, References
% ТЕКУЩАЯ РЕАЛИЗАЦИЯ: отсутствует

\end{document}