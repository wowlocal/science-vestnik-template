% ========================================
% КЛАСС ДОКУМЕНТА
% ========================================
% Требование: кегль 12 для основного текста
\documentclass[12pt]{article}    % article - класс для статей, 12pt - размер шрифта

% ========================================
% ПОДДЕРЖКА РУССКОГО ЯЗЫКА
% ========================================
% Требование: поддержка русского языка для текста статьи
\usepackage[utf8]{inputenc}      % Кодировка UTF-8 для входных файлов
\usepackage[T2A]{fontenc}        % Кодировка T2A для кириллических шрифтов
\usepackage[russian]{babel}      % Правила переносов и локализация для русского языка

% ========================================
% ПОЛЯ СТРАНИЦЫ
% ========================================
% Требование: настройка полей документа
\usepackage{geometry}
\geometry{margin=1in}            % Поля 1 дюйм (2.54 см) со всех сторон

% ========================================
% ШРИФТ ОСНОВНОГО ТЕКСТА
% ========================================
% Требование: Times New Roman, кегль 12, интервал 1,5
\usepackage{newtxtext}           % Шрифт Times New Roman для основного текста

% ========================================
% МЕЖСТРОЧНЫЙ ИНТЕРВАЛ
% ========================================
% Требование: интервал 1,5 для основного текста
\usepackage{setspace}
\onehalfspacing                  % Устанавливает межстрочный интервал 1.5

% ========================================
% АБЗАЦНЫЙ ОТСТУП
% ========================================
% Требование: абзацный отступ 1,25 см
\setlength{\parindent}{1.25cm}   % Устанавливает отступ первой строки абзаца 1.25 см

% ========================================
% НАСТРОЙКА СНОСОК
% ========================================
% Требование: Times New Roman 10pt, интервал 1
\usepackage[bottom]{footmisc}    % Размещает сноски внизу страницы
% Настройка размера шрифта сносок (10pt) и одинарного интервала
\makeatletter
\renewcommand\@makefntext[1]{%
    \setlength{\parindent}{1.25cm}%  % Сохраняем абзацный отступ в сносках
    \noindent
    \makebox[1.8em][l]{\@thefnmark}% Номер сноски
    \fontsize{10}{12}\selectfont     % Шрифт 10pt с интервалом 1.0 (12pt базовая линия)
    #1%
}
\makeatother

% ========================================
% НУМЕРАЦИЯ СТРАНИЦ
% ========================================
% Требование: все страницы рукописи нумеруются, включая список литературы
\pagestyle{plain}                % Стиль страницы с номером внизу по центру

% ========================================
% МАКРОСЫ ДЛЯ СТРУКТУРЫ СТАТЬИ
% ========================================
% Требование: создание всех обязательных элементов статьи по требованиям журнала

% УДК (универсальная десятичная классификация)
\newcommand{\udc}[1]{%
    \noindent\textbf{УДК #1}%
    \par\vspace{0.5em}%
}

% Заголовок статьи на русском (полужирный, прописные, 14pt)
\newcommand{\articletitleru}[1]{%
    \begin{center}
        \fontsize{14}{16}\selectfont\bfseries\MakeUppercase{#1}
    \end{center}
    \vspace{0.3em}
}

% Автор(ы) на русском (полужирный, 14pt)
\newcommand{\authorru}[1]{%
    \begin{center}
        \fontsize{14}{16}\selectfont\bfseries #1
    \end{center}
    \vspace{0.3em}
}

% Организация и адрес (курсив, 12pt)
\newcommand{\institution}[1]{%
    \begin{center}
        \fontsize{12}{14}\selectfont\itshape #1
    \end{center}
    \vspace{0.5em}
}

% Аннотация на русском (полужирный + курсив, 200-300 слов)
\newcommand{\abstractru}[1]{%
    \noindent\textbf{\textit{Аннотация:}} \textit{#1}%
    \par\vspace{0.5em}%
}

% Ключевые слова на русском (полужирный + курсив, 5-7 слов)
\newcommand{\keywordsru}[1]{%
    \noindent\textbf{\textit{Ключевые слова:}} \textit{#1}%
    \par\vspace{0.5em}%
}

% Финансирование (опционально)
\newcommand{\funding}[1]{%
    \noindent\textbf{Финансирование:} #1%
    \par\vspace{0.5em}%
}

% Блок "Для цитирования"
\newcommand{\citation}[1]{%
    \noindent\textbf{Для цитирования:} #1%
    \par\vspace{1em}%
}

% ========================================
% ИНФОРМАЦИЯ О ДОКУМЕНТЕ (Пример заполнения)
% ========================================
% Здесь указываются данные для конкретной статьи
% После заполнения эти команды будут использованы в \maketitle

\title{Название статьи}
\author{Имя автора}
\date{\today}

\begin{document}

% ========================================
% ЗАГОЛОВОЧНАЯ ЧАСТЬ СТАТЬИ (РУССКИЙ)
% ========================================

% УДК - укажите код классификации
\udc{000.00}

% Заголовок статьи на русском (полужирный, прописные, 14pt)
\articletitleru{Название статьи}

% ФИО автора(ов) на русском (полужирный, 14pt)
\authorru{И.О. Фамилия\footnote{%
    \textbf{Фамилия Имя Отчество} — кандидат наук, должность, Название организации. Email: example@email.com%
}}

% Организация и адрес (курсив, 12pt)
\institution{Название организации, адрес организации}

% Аннотация (полужирный + курсив, 200-300 слов)
\abstractru{Здесь размещается аннотация объемом 200--300 слов. Аннотация должна кратко излагать содержание статьи, основные результаты исследования и выводы. Текст аннотации должен быть структурированным и информативным, позволяя читателю понять суть работы без обращения к полному тексту.}

% Ключевые слова (полужирный + курсив, 5-7 слов)
\keywordsru{ключевое слово 1, ключевое слово 2, ключевое слово 3, ключевое слово 4, ключевое слово 5}

% Финансирование (опционально, раскомментируйте при необходимости)
% \funding{Исследование выполнено при поддержке гранта...}

% Для цитирования с DOI
\citation{Фамилия И.О. Название статьи // Вестн. Моск. ун-та. Сер. 13. Востоковедение. 2025. Т. XX, № X. С. 00--00. DOI: 10.55959/MSU0320-8095-13-00-0-0.}

\vspace{1em}

% ========================================
% ОСНОВНОЙ ТЕКСТ СТАТЬИ
% ========================================
% Требование: разделы с подзаголовками (допускаются)
% Кегль 12, Times New Roman, интервал 1.5

\section{Введение}

Текст введения.

\section{Основная часть}

Основной текст статьи.

\subsection{Подраздел}

Более детальное содержание в подразделе.

\section{Заключение}

Текст заключения.

% ========================================
% СПИСОК ЛИТЕРАТУРЫ
% ========================================
% Требование: подстрочные примечания + список литературы (полужирный заголовок)
% Порядок: кириллица → латиница → иероглифика (алфавитный внутри каждой группы)
% ТЕКУЩАЯ РЕАЛИЗАЦИЯ: отсутствует

% ========================================
% АНГЛИЙСКАЯ ВЕРСИЯ
% ========================================
% Требование: Title, Author, Institution, Abstract, Keywords, References
% ТЕКУЩАЯ РЕАЛИЗАЦИЯ: отсутствует

\end{document}